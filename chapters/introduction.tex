\chapter{Introduction}
\label{chap:intro}
% bit informal introduction
Assume you are in your first year at the Belgium University of Ghent. Like most university students, you live in a dorm, and unless you lived in Ghent before moving, you do not know your way in Ghent. What is the best restaurant, and what can you do for fun? But at some point, you must go to your course, take an exam, or take a break from student life at your parents' home. Being late is not an option in these situations. But as a first-year and carless, you wonder how you will reach these destinations.  

The most obvious would be \glsxtrfull{pt}. Using \glsxtrshort{pt} is economical as a car costs ten times more yearly. Further, it is one of the few transportation modes that is environmentally friendly.

Now you know how you can reach those destinations. You question which route is best to take. Further, you want to be sure if you will arrive on time and at which time you should leave. 
% Intro route planners
For this last problem, there is a high chance that you will use a route planner. Many route planners exist, like \url{nmbs.be}, \url{delijn.be} or \url{maps.google.com}. Even open-source solutions exist, like Open Trip Planner\cite{noauthor_otp_2023}.

Underlying these route planners, an algorithm calculates the "best" route serving your search criteria. They produce Pareto-optimal journeys. A journey is defined as a sequence of trips between two points. A trip can be made using a variety of vehicles (train, bus, etc.) or even on foot. One thing to remark is that if a journey contains $K$ trips, they are $K-1$ transfers. 

Commonly, these algorithms are graph-based. Although known speedup techniques can achieve up to several million on route planners, these speedups are generally more focused on road networks \cite{bast_car_2009}. When looking at a more specific domain, like \glsxtrshort{pt}. A few difficulties arise.

% difficulties road planners and PT
To start, \glsxtrshort{pt} has a different structure than for road networks. For example, the travel times are enough as criteria to compute a road journey, but for \glsxtrshort{pt}, additional criteria could be the number of transfers or the cost of the different transportation modes (train, bus, metro,...). These extra criteria add more complexity to the calculations.

Footpaths are a significant influence on the resulting Pareto journeys. It could be more efficient to walk to a different bus stop. However, unrestricted footpaths would cause too many edges in graph-based algorithms.

% centralized data strategy
Another difficulty is the centralized data strategy used by most route planners \cite{rojas_melendez_julian_andres_decentralized_2020}:
\begin{enumerate}
    \item For each transit mode, collect a dataset. A widely used format for exchanging a PT dataset is GTFS \cite{noauthor_gtfs_2022}.
    \item Integrate the collected datasets using a predefined data model in a centralized data store.
    \item Calculates available routes using a route planning algorithm tailored to run over the predefined data model.
\end{enumerate}
% problems of centralized strategy
The centralized strategy causes applications to be unscalable and results in high computational infrastructure costs (total cost of ownership). 

% work in progress
Another problem is that the user cannot use the dataset in his queries if the central system does not support a third-party dataset. The data needs to be homogenous to integrate it into the route planner. If the dataset is heterogeneous, integration is done manually, and the route planner algorithm must be adapted.

\section{Goal}

\begin{figure}[H]
\centering
\resizebox{\textwidth}{!}{
\begin{tikzpicture}

\draw [stealth-stealth](0,0) -- (10,0);

\draw (10,0) node[above=32pt,right=-2.5 cm] {2013: Connection Scan Algorithm};

\end{tikzpicture}}
    \caption{This figure represents what we want to accomplish a solution that not only runs on the client and not only on the server.}
    \label{fig:shared}
\end{figure}

We want to avoid the central data strategy and let the client calculate the ideal route. However, we do not want to create a fully offline device since data will quickly become outdated and unusable. Data must be sent to the client from the server, ideally with little or no processing. 

An extra challenge is to decide which data is relevant for the client to send before the client needs it to save bandwidth and keep the number of requests low. If we send too much data, the client can become slow to high load on received.
 
Goals:
\begin{enumerate}
    \item Create a shared responsibility: The client calculates, and the server manages data.
    \item Experiment with a fragmentation to only send the required data to the client.
    \item Use an ontology to support multiple datasets/sources.
\end{enumerate}
