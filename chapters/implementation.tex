\chapter{Implementation}
\label{chap:implementation}

Initially, the implementation was divided into two parts. We discuss the implementation of \glsxtrshort{raptor} for the browser in \autoref{section:implementation_raptor}. The conversion and download of GTFS data is discussed in \autoref{section:implementation_data_ontology}.


\section{\glsfmtfull{raptor} implementation}\label{section:implementation_raptor}
\subsection{Data usage ofexistingimplemtation}
Since we were inspired by an existing implementation of Raptor, we did some research on which data the parser/ implementation used.

Small description of text files used and how they are transformed.
$
- link = ? (Think footwalks) => Object "transfers" with ke from_stop_id (=origin) -> list
- calendar = calendar.txt => Object "calendars", uses service_id as key
- calendar_date = calendar_dates.txt => Object "dates" with key service id -> object with key date -> row.exception_type === "1" (service exceptions)
- trip = trips.txt => Array "trips"
- stop_time = stop_times.txt => Object "StopTimes" with key_trip id -> list
- transfer = transfers.txt => Object "Interchange" for transfer from and to the same stop. || Object "transfers" with key from_stop_id (=origin) -> list
- stop = stops.txt => Object "stops" with key stop_id
$
\subsubsection{Not used but are in our \glsxtrshort{gtfs}-feed}
\begin{itemize}
    \item feed\_info: information the feed itself.
    \item agency: information about the agency providing the services.
    \item areas: 
    \item translations:
\end{itemize}
\subsection{Unavailable in feed}
\begin{itemize}
    \item shapes (shape of a route, handy for visualizations)
    \item fare\_rules
    \item fare\_attributes
    \item levels
    \item attributions
\end{itemize}

\subsection{small mistakes of the implementation}
\subsubsection{unused information from gtfsfeed}
The algorithm did not use some information but was thereby inferred. For example:
\begin{itemize}
    \item Order of stop points. Each stop point in a sequence was based on its position in the array. Which is not a big problem if the array is correctly sorted. But a \glsxtrshort{gtfs} feed requires each stop point to carry information like their position in the sequence. This is more correct since it supports feeds with incorrectly sorted stop points.
    \item Route id was not used but created from the trip\_id 
\end{itemize}
\section{GTFS downloader}
Since a GTFS feed is a zip file valid for a certain period, we need a small program to manage all these versions. This led to a Node.js module gtfs-downloader\footnote{Available as GitLab project under the same name}. It provides the file path of the GTFS zip archive that either has been downloaded in the past or is freshly downloaded. So, it also acts like a cache and is more important than it appears, as only the most recent is available on the server. For debugging and experimenting while developing, it is handy that we have some control over the version in use. 

Two sets of functionality are represented in \autoref{fig:gtfsdownloader}. The first is when we already know the version we want to download. We search if it exists in our data folder; if not, we check if it is available on the server for download.

In the other case, we want to have the latest version, and we send a head request to the server\footnote{Server URL is specified in a dot env config} since we are only interested in the LastModified header. We check if that version is available locally; otherwise, we download the zip file.
\makeatletter
\begin{figure}[H]
\centering
\resizebox{.8\textwidth}{!}{%
    \begin{tikzpicture}[
        database/.style={
        path picture={
            \draw (0, 1.5*\database@segmentheight) circle [x radius=\database@radius,y radius=\database@aspectratio*\database@radius];
            \draw (-\database@radius, 0.5*\database@segmentheight) arc [start angle=180,end angle=360,x radius=\database@radius, y radius=\database@aspectratio*\database@radius];
            \draw (-\database@radius,-0.5*\database@segmentheight) arc [start angle=180,end angle=360,x radius=\database@radius, y radius=\database@aspectratio*\database@radius];
            \draw (-\database@radius,1.5*\database@segmentheight) -- ++(0,-3*\database@segmentheight) arc [start angle=180,end angle=360,x radius=\database@radius, y radius=\database@aspectratio*\database@radius] -- ++(0,3*\database@segmentheight);
        },
        minimum width=2*\database@radius + \pgflinewidth,
        minimum height=3*\database@segmentheight + 2*\database@aspectratio*\database@radius + \pgflinewidth,
    },
    database segment height/.store in=\database@segmentheight,
    database radius/.store in=\database@radius,
    database aspect ratio/.store in=\database@aspectratio,
    database segment height=0.1cm,
    database radius=0.25cm,
    database aspect ratio=0.35,
    roundnode/.style={circle, thick, minimum size=3em,align = center},
green/.style={draw=green!60, fill=green!5},
orange/.style={draw=orange!60, fill=orange!5},
white/.style={draw=black!60, fill=white},
    files/.style={double copy shadow,
        fill=white,
        draw=black,
        minimum width=5em,
        minimum height=2em}
    ]
        
    % draw nodes
    \node[files] (local) at (0,0) {local data};
    \node[roundnode,white] (program) at (8,0)  {GTFS downloader,\\ get specific version};
    \node[database, label=below:server gtfs.be] (server) at (15,0) {};

        % draw nodes
    \node[files] (local2) at (0,-5) {local data};
    \node[roundnode,white] (program2) at (8,-5)  {GTFS downloader,\\ get latest};
    \node[database, label=below:server gtfs.be] (server2) at (15,-5) {};
    % Draw edges
    \draw (local) edge[<-, bend left=15] node[pos=0.5,above] {1) check if local version exist} (program);
    \draw (local) edge[->,bend right=15] node[pos=0.5,below] {2) give filepath on existence} (program);
    \draw (program) edge[->,bend left=15] node[pos=0.5,above] {3) else check the server version} (server);
    \draw (program) edge[<-,bend right=15] node[pos=0.5,below] {4) Download file if  lastmodified and version match} (server);


    \draw (program2) edge[->, bend left=15] node[pos=0.5,above] {1) Head request} (server2);
    \draw (program2) edge[<-,bend right=15] node[pos=0.5,below] {2) headers of file: Lastmodified} (server2);
    \draw (program2) edge[->,bend right=15] node[pos=0.5,above] {3) check if latest version already exists} (local2);
    \draw (program2) edge[->,bend left=15] node[pos=0.5,below] {4) give filepath on existence} (local2);
    \draw (program2) edge[->,bend left=50] node[pos=0.5,above] {5) Download if not local} (server2);

    \end{tikzpicture}
    }
    \caption{Interaction model of the gtfs downloader.}
    \label{fig:gtfsdownloader}
\end{figure}
\makeatother
\section{Data parser/ontology}\label{section:implementation_data_ontology}
Since most datasets, including the datasets of the \glsxtrshort{nmbs} we use, use the \glsxtrshort{gtfs} Data model, we decided to write a program to convert \glsxtrshort{gtfs} to the \glsxtrshort{oslo} ontology.
\subsection{Ontology}
\begin{figure}[H]
\resizebox{\textwidth}{!}{%
    \begin{tikzpicture}

\umlclass[x=0,y=0]{Line}{
  name : string \\ short\_name : string
}{}
\umlclass[x=6,y=0]{Route}{
  name : string
}{}

% connect line and route
\umlassoc[geometry=-|-, arg1=Line, mult1=0..1, pos1=0.3,  arg2=ConsistsOf, mult2=0..*, pos2=2, align2=left]{Line}{Route}

\umlclass[x=13,y=0]{Service Joruney pattern}{}{}
\umlassoc[geometry=-|-, arg1=Route, mult1=0..1, pos1=0.3, arg2=CoveredBy, mult2=0..*, pos2=1, align2=left]{Route}{Service Joruney pattern}
\umlclass[x=0,y=-6]{Planned Stop Point}{}{}
\umlclass[x=8,y=-3]{Passing Time}{}{}




%\umlnote[x=2.5,y=-6, width=3cm]{Route}{Je suis une note qui concerne la classe B}

    \end{tikzpicture}
    }
    \caption{Timeline describing the developments in route planning. The crossed parts represent}
    \label{fig:ontology}
\end{figure}
\subsection{Parser}
The initial plan was to reuse the parser, but it used a package called gtfs-stream \cite{noauthor_staecogtfs-stream_2024}. The module takes a Readable stream object of a \glsxtrshort{gtfs} zip file and creates a pipe stream. The stream provides you  with an object with one of the following types: String, feed\_info, agency, stop, route, trip, stop\_time, calendar, calendar\_date, fare\_attribute, fare\_rule, shape, frequency, transfer)

This package is excellent if you want to read a \glsxtrshort{gtfs}-feed, but the package has one significant shortcoming for converting a feed to a different ontology. Defining the order in which the module reads the files is impossible. For example, this is not possible if you want to read about the trips before stop times.

We chose to implement something similar but without the gtfs-stream module. 
\subsubsection{Node streams}
First, we explain NodeJS streams. A stream is an abstract interface for working with streaming data in Node.js. \cite{noauthor_stream_nodate} They are similar to arrays and other collections. But what makes them powerful is that not all data is available, so they do not have to fit in memory all at once. Especially when working with big data sets.

They are either readable or writeable.

\subsubsection{our \glsfmtshort{gtfs} streamer}
To have more control over how to read files of a \glsxtrshort{gtfs} feed, we implemented our own \glsxtrshort{gtfs} streamer with two modules.



The first module is a zip reader called node-stream-zip \cite{noauthor_node-stream-zip_2021}. Like the \glsxtrshort{gtfs}-stream package, it creates a readable stream object and gives back a stream. A real improvement is that we can provide a filename that we want to extract from the zip file. It is unnecessary to read the full \glsxtrshort{gtfs} file. This offers excellent flexibility, solves the problem of ordering and gives immediate support for new files (if added in the \glsxtrshort{gtfs} specification) or out-of-spec files.

Now, we can read the files in the order we want and parse every line. Since \glsxtrshort{gtfs} is stored as \glsxtrshort{csv}, we can use any \glsxtrshort{csv} parser. We opted for csv-parse \cite{noauthor_csv-parse_2024}, a popular, well-known module. We give the parser the delimiter (",") and the columns option. The columns have the effect that the rows are returned as \glsxtrshort{json} objects instead of arrays.

\begin{listing}[H]
\inputminted[linenos,frame=single,breaklines]{TypeScript}{code/own_gtfs_streamer.js}
    \caption{Our own version of \glsfmtshort{gtfs}-streamer}
\label{listing:gtfs:streamer}
\end{listing}


\subsubsection{MongoDB}
After reading the \glsxtrshort{csv} files correctly, we are left with converting the \glsxtrshort{json} object to the ontology and inserting them in MongoDB.

Some difficulties directly arose. We want to have a merged view in MongoDB. We did not want all entities in different collections. So, we had to insert the planned stop points and then update them when extra information became available. At first, we used the updateOne API from Mongodb. 
\begin{listing}[H]
    \begin{minted}[linenos,frame=single]{TypeScript}
let collection = connection.db(db_name).collection(collection_name);
return collection.updateOne(id, update);
    \end{minted}
    \caption{Usage of updateOne in mongoDB}
\label{listing:mongoDB:updateone}
\end{listing}
The problem with that is the combination of updateOne in a NodeJS stream. Quickly, the memory begins to rise, eventually crashing the program due to a lack of system resources.

To avoid memory problems, we let each transformation function for a row of \glsxtrshort{csv} return a MongoDB write operation. These write operations can be given to bulkWrite of MongoDB. 

There are six types Of write operations:
\begin{enumerate}
    \item UpdateOne
    \item UpdateMany
    \item InsertOne
    \item DeleteOne
    \item DeleteMany
    \item ReplaceOne
\end{enumerate}

We further leverage Typescript to define a generic class to ease the programming burden. 

\begin{listing}[H]
    \inputminted[linenos,frame=single,breaklines]{TypeScript}{code/type_write_op.ts}
    \caption{A generic class with a constructor can easily create write operations.}
\end{listing}

\begin{listing}[H]
    \inputminted[linenos,frame=single,breaklines]{TypeScript}{code/type_write_op_instan.ts}
    \caption{An example of a possible instantiation of a generic class.}
\end{listing}

\subsection{comformity of ontology}
\subsubsection{comformity application profiles}
\glsxtrshort{oslo} has also described when a \glsxtrshort{jsonld} document conforms to an application profile \cite{noauthor_conformiteit_nodate}.

An application profile is a specification for data exchange that introduces additional constraints for applying vocabularia. These constraints could include refinement of terminology (classes and properties) consistent with the semantics from the relevant specifications with a well-defined usage as a goal;
    External terminology (classes and properties) is used for new/extra terms not found in the existing vocabulary.

To conform with the \glsxtrshort{oslo} application profile, the following constraints have to be met:
\begin{itemize}
    \item \textbf{Must} each class contains the attributes that have a cardinality of one/
    \item \textbf{Forbidden} for a class attribute with a cardinality of maximum one to have more instantiations.
    \item \textbf{Forbidden} to use terminology of vocabularia not defined in the application profile.
    \item \textbf{Allowed} to use terminology in a way that is consistent with her semantics (definition, use, domain and range)
    \item \textbf{Allowed} to extend with other vocabularies that do \textbf{do not overlap} with terminology from this vocabulary. 
\end{itemize}
\subsection{Nesting}
\subsection{Result}
\subsection{Mistakes in context}
When we first downloaded the ontology, we noticed it wouldn't work in the \glsxtrshort{jsonld} playground. Upon closer inspection, we noticed 
$$@type; ""$$
\section{Bringing two worlds together: experimentation of fragmenting}
%\subsection{first naive idea}
